%!TEX root = ../TTK4550-MHT.tex
\section{Introduction}
\subsection{Motivation}
This report is the result of TTK4550 Engineering cybernetics specialization report. Where the aim is to \emph{let the student specialize in a selected area based on scientific methods, collect supplementary information based on literature search and other sources and combine this with own knowledge into a project report}. The learning outcome of this project is to give the student an extensive knowledge in a current problem, good knowledge in related topics and relevant scientific literature.

\subsection{Autosea}
This project is a part of the Sensor fusion and collision avoidance for autonomous surface vessels (Autosea) project, which is a collaboration between NTNU AMOS and Maritime Robotics, DNV GL and Kongsberg Maritime. The vision for the project is to \emph{attain world-leading competence and knowledge in the design and verification of methods and system for sensor fusion and collision avoidance for autonomous surface vessels}.

\subsection{Problem description}
The proposed title for this project (and the following master-thesis) was "Multi-target tracking using radar and AIS". The key idea behind this formulation is that radar and AIS have different strengths and weaknesses, and if utilized properly, the strengths of both system can be exploited, while the weaknesses can be reduced. The following task was proposed for this project:
\begin{itemize}
\item Write a survey on multi-target and multi-sensor tracking methods.
\item Implement a multi-target tracking method that fuses radar data with AIS data under benign assumptions.
\item Describe the method and summarize the findings in a report.
\end{itemize}

\subsection{Goals}
The goals for this project is to implement a radar tracking algorithm in Python based on one of the surveyed methods, test this with simulated and real data and analyze the performance for different conditions and discuss methods for improvement.

\subsection{Outline of report}
In section \ref{sec:survey}, different methods for target tracking and data association are presented in varying depth, more on the most relevant methods and an overview on the others. In section \ref{sec:algorithm}, the algorithm based on the selected approach is thoroughly explained, and in section \ref{sec:ilp} the integer optimization problem that arises in section \ref{sec:algorithm} is elaborated. The results are presented in section \ref{sec:results} and discussed in section \ref{sec:discussion}.
