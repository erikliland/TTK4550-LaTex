%!TEX root = ../TTK4550-MHT.tex
\section{Closing remarks}
\subsection{Conclusion}
This report has shown that off-the-shelf \gls{ilp} \glspl{solver}, both free and commercial, are capable of solving the data association problem in a \gls{tomht}. It has also provided evidence that the \gls{tracking} performance does benefit very little from more than $N=3$ scan history in moderate manoeuvring scenarios when \gls{Pd} $> 70\%$, even with severe amounts of \gls{clutter}.

Through simulations on a single-core implementation of a \gls{tomht} in \gls{python}, it have also shown the feasibility for real-time applications with five closely spaced targets with $N=5$ in a heavy clutter scenario ($\lambda_\phi = 1 \cdot 10^{-3}$), and $N=6$ for a moderate clutter scenario ($\lambda_\phi = 5 \cdot 10 ^{-4}$). Further, this report has  outlined the possibilities for significant rune-time improvements through the use of multi-core \glspl{cpu} and \glspl{gpu}.

\subsection{Future work}
More complex \gls{pruning} and merging of hypotheses is an area where there might be substantial benefits to both performance and run time, since it allows for longer history. By nature, many hypotheses will have very similar paths (see Figure \ref{fig:track_hypotheses_example}) and could therefore be candidates for being merged. The challenge with this is that the assignment problem could be rendered infeasible and crash the algorithm. A novel approach could be to formulate the pruning as an optimization problem. With this approach it could be possible to also remove hypotheses with a low score, subject to the problem still being feasible.

In the maritime context, it could be very beneficial to use the maritime \gls{ais} to aid the \gls{tracking}. There are different approaches to include this extra set of data, all with their pros and cons. One option could be to use the \gls{ais} track to weight the measurements from the radar; this way the measurement that are more likely to originate from a target is given a higher score. Another possibility is to do track-to-track fusion / filtering, where the best or N-best tracks from the radar are filtered with the tracks from the AIS system. Track-to-track fusion can also be handled within the ILP framework, see \cite{Coraluppi2000}.

A final but very important part, required for the system to be complete, is track initialization. As with \gls{ais} assisted \gls{tracking}, there are plenty of ways to tackle this challenge, some of them mentioned in section \ref{subsec:tomht}.