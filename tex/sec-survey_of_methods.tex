
%!TEX root = ../TTK4550-MHT.tex
\section{Survey of multi-target tracking methods}
\label{sec:survey}
The aim of this section is to give the reader a brief overview of tracking as a problem and a feeling for the most popular methods, their assumptions and strong and weak properties as seen from an 2D maritime anti collision perspective.

\subsection{Tracking}
Tracking of an object (target) is the process of estimating its state (i.e. position and velocity) based on discrete measurements from an observation system. An observation system can be a radar, sonar or any other sensor that passively or actively detects objects within an area or volume.

\subsection{Tracking system}
A tracking system can be interpreted as either the complete system from the signal processing level to the finished tracks, or as I define it in this text: \emph{A system that process consecutive measurements from an observation system and collects measurements from the same target into tracks or initiate new tracks.} A track is a subset of all the measurements from the observation system that is believed to originate from the same target. The challenge knowing which measurement originating from which (real) target is the core at any tracking system. This association problem is non-trivial even under ideal situations, and the addition of spurious measurements and missed targets only increases the complexity.

There has been developed a large variety of methods to overcome this association problem, and most of them have several sub-methods. In the following subsections, some of the most common and popular methods will be presented.

\subsection{Nearest Neighbour Filter}
The Nearest Neighbour Filter (NNF) is the simplest approach in tracking, where one always select the closes neighbour as the consecutive measurement in the track. This approach suffers from being very vulnerable to clutter and dense target scenarios. It can be somewhat improved by estimating an a-priori state through a Kalman Filter and selecting the nearest neighbour to the estimate. This extension is sometimes refereed to as Nearest Neighbour Standard Filter (NNSF). Under the (normal) assumption that each target can at maximum generate one measurement, the NNF and NNSF are both single-target methods in the way that they will assign the same measurement to more that one track. They can however be expanded to multi-target variants by formulating the problem as a global least squares integer optimization problem, often called Global Nearest Neighbour Filter (GNNF). With this extension, the NNSF is almost becoming a zero-scan multi hypothesis tracker. NNF, NNSF and GNNF are the only method presented which is non-probabilistic, and does not assume specific models for noise, clutter, false alarm rate or similar.

\subsection{Probabilistic Data Association Filter}
Probabilistic Data Association Filter (PDAF) is a \emph{single-target} Bayesian association filter which is based on single scan probabilistic analysis of measurements. At each scan the filter is calculating a most probable measurement based on a weighted sum on the measurement innovations inside a validation region.
\begin{equation}
\V{\hat{y}} \triangleq \sum\limits_{j=1}^m \beta_j \V{ \tilde{y}_j }
\label{PDA_sum}
\end{equation} 
where
\begin{equation*}
\begin{split}
	\beta_j		&= \text{the probability of measurement j to be the correct one} \\
	\tilde{y}_j &= y_j - \hat{y}_j \hspace{5mm}	\text{the j-th measurement innovation}
\end{split}
\end{equation*}
PDAF is computationally modest (approximate 50\% more computationally demanding than a Kalman Filter \cite{Bar-Shalom1998} p.163) and have good results in an environment with up to about 5 false measurement in a $4\sigma$ validation region \cite{Bar-Shalom1998}. PDAF does not include track initialization and assumes that at most one measurement can originate from an actual target. It also assumes that clutter is uniformly distributed in the measurement space and that the targets history is approximated by a Gaussian with a calculated mean and covariance (single scan).

PDAF can be used for multiple targets, but only as multiple copies of the single-target filter \cite{Fortmann1983}. Since PDAF is generating a "best guess"-measurement from all the measurements inside its validation region, it can suffer from track coalescence. This coalescence occurs when two targets have similar paths, and the resulting tracks will be an "average" of the two (actual) tracks. There has been done some work to overcome this coalescence \cite{Blom2000}.


\subsection{Joint Probabilistic Data Association Filter}
Joint Probabilistic Data Association Filter (JPDAF) is a \emph{multi-target} extension of the Probabilistic Data Association Filter in which joint posteriori association probabilities are calculated for every target at each scan. Both PDAF and JPDAF use the same weighted sum (\ref{PDA_sum}), the key difference is the way the weight $ \beta_j $ is calculated. Whereas PDAF treats all but one measurement inside its validation region as clutter, in JPDAF the targets which interacts (one cluster) are treated as connected and the connected $ \beta_j $s are computed jointly across the cluster set with a given set of active targets inside the cluster. The probability of a measurement $j$ belonging to a target
$ t $ is \cite{Fortmann1983}
\begin{equation}
\begin{split}
\beta_j^t &= \sum_{\chi} P\{ \chi | Y^k \} \hat{\omega}_{jt}(\chi) \\
\beta_0^t &= 1 - \sum_{j=1}^m \beta_j^t
\end{split}
\end{equation}
where
\begin{equation}
P\{\chi|Y^k\} = \frac{C^\phi}{c} 
				\prod_{j:\tau_j=1} \frac{exp[-\frac{1}{2}(\V{\tilde{y}_j^{t_j}})^T S_{t_j}^{-1}(\V{\tilde{y}^{t_j}})]}{(2\pi)^{M/2} |S_{t_j}|^{1/2}}
				\prod_{t:\delta_t=1} P_D^t
				\prod_{t:\delta_t=0} (1 - P_D^t).
\end{equation}
Where 
\begin{equation*}
\begin{split}
	\beta_j^k			&= \text{the probability that measurement j belongs to target k} \\
	\beta_0^k 			&= \text{the probability that no measurement belongs to target k} \\
	\chi 				&= \text{All feasible events} \\
	Y^k 				&= \text{all candidate measurements up to and included time k} \\
	\hat{\omega}_{jt}	&=	\begin{cases}
								1, \text{if } \chi_{jt} \text{ occurs} \\ 
								0, \text{otherwise} 
							\end{cases} \\
	m					&= \text{number of measurements}
\end{split}
\end{equation*}
Since the JPDAF is calculating joint probabilities for all the combinations of measurement associations in the cluster, the computation demand is growing exponentially with the numbers of tracks and measurements in the cluster. A real time implementation of the JPDAF has been developed and patented by QinetiQ \cite{QinetiQ2003}, and described in \cite{Horridge}.

JPDAF also have a weakness when the targets are closely spaced, were it tends to merge targets. This coalescence is a result of the filter considering the average of two tracks more likely than two separate tracks. Seen from a anti collision safety perspective is coalescence not acceptable. An improvement to the JPDAF has been proposed by \cite{Blom2000}.

Since JPDAF is using all the measurements inside the gate for each track, one measurement can be used to update more than one track if is within more that one gate.

\subsection{Multi Hypothesis Tracker}
Multiple hypothesis tracking (MHT) is a decision logic which generates alternative hypotheses when new measurement are received and within the gate. By making several possible hypotheses, the decision in which measurement to choose can be propagated into the future when more information is available. Each hypothesis is given a score or probability as a measure of the goodness of the measurement, which are accumulated to evaluate the combinations of consecutive measurements.

In contrast to PDA methods which in some cases will estimate an "average" of two tracks as the true one (coalesce), MHT methods split when in doubt. The idea of using multiple hypotheses was first introduced by Singer. et al. \cite{Singer1974}, but the first complete algorithm was presented by Reid \cite{Reid1978}, where a hypothesis oriented MHT was developed. Following this, a track oriented MHT was proposed in \cite{Kurien1990} and improved by \cite{Bar-Shalom2007}. MHT was developed under the assumption that at most one measurement can originate from each target in each scan, and that a target does not necessary show on every scan (Probability of detection, $P_D < 1$).

Since MHT always selects the best tracks at any time, this can cause apparent inconsistency in the output to the user whenever a new set of measurements are received. If this is unwanted, an alternative representation is to output the N-best tracks and display them to the user in a way that visualises the probability. A third option is to output a weighted average of the tracks along with the covariance of the average.

The MHT approach to tracking and data association was for a long time dismissed because of its computationally large cost. The dramatic increase in computational capability from the 1980´s to the late 2010´s have however lead to a new spring for MHT, and have led to a near universal acceptance as the preferred data association method in modern tracking system \cite{Blackman2004}. Already in 2001 did Blackman publish a demonstration that MHT is capable of real-time demands \cite{Blackman2001}

There are two main approaches to MHT, hypothesis oriented and track oriented.

\subsubsection{Hypothesis Oriented MHT}
Hypothesis Oriented MHT (HOMHT) or Measurement Oriented MHT (MOMHT) is a fully Bayesian approach where direct probabilities of global joint measurement-to-target association hypothesis are calculated. The algorithm initiates tracks and handles missing measurements, it has a recursive nature and allows for clustering for quicker computation.
 
When a new set of measurements is received, Reid´s method is defining a set of hypotheses, each containing a complete set of associations of the existing tracks and new measurements. By defining the hypotheses in this way, they become compatible in the sense that only one can and need to be selected from that time step. %example?
When the next set of measurements arrives, each of the current hypotheses are expanded with all measurement-to-track assignments for the new measurements. This way, the hypotheses will keep their compatibility.

When evaluating the alternative hypotheses, each is assigned a probability. This probability takes into account the false-alarm statistics of the measurement system, the expected density of targets and clutter and the accuracy of the target estimates. The probability of each data association hypothesis was developed by Reid in \cite{Reid1978}.
\begin{equation}
P_i^k = \frac{1}{c} P_D^{N_{DT}}(1-P_D)^{(N_{TGT}-N_{DT})} \beta_{FT}^{N_{FT}} \beta_{NT}^{N_{NT}} \left[ \prod_{m=1}^{N_{DT}} N(\V{Z_m}-\M{H}\V{\bar{x}},\M{B}) \right] P_g^{k-1}
\end{equation}
where 
\begin{equation*}
\begin{split}
	P_i^k		&= \text{the probability of hypothesis $\Omega_i^k$ given measurements up through time $k$} \\
	P_D 		&= \text{the probability of detection} \\
	\beta_{FT} 	&= \text{the density of targets} \\
	\beta_{NT}	&= \text{the density of previously unknown targets that have been detected} \\
	N_{DT} 		&=	\text{number of designated target} \\
	N_{FT} 		&= \text{the number of false targets} \\
	N_{NT} 		&= \text{the number of new targets} \\
	N_{TGT} 	&= \text{is the number of targets} \\
	\V{Z_m} 	&= \text{the m-th measurement in the current scan} \\
	\M{H} 		&= \text{the observation matrix} \\
	\M{B} 		&= \text{the measurement covariance}
\end{split}
\end{equation*}

\subsubsection{Track Oriented MHT}
Track Oriented MHT (TOMHT) is a "bottom-up" approach where the tracks are assumed initialized, and for each scan the track splits whenever there are more than one feasible measurement in the validation region (in addition to the no-measurement hypothesis). In comparison to HOMHT, a hypothesis is now defined as a possible track for single target, and multiple hypotheses need to be combined to get a global hypothesis. If the tracks have no measurement in common, the optimal global hypothesis is the union of the best hypothesis for each target. Whenever a measurement are assigned to two or more targets, the problem of selecting the optimal combination of tracks becomes an N-dimensional assignment problem.

New track hypothesis is generated from the filtered estimate from a Kalman Filter, and a score is calculated using (20) from \cite{Bar-Shalom2007}. S. Blackman is arguing in \cite{Blackman2004}, that modern tracking system should use an Interacting Multiple Model (IMM) approach in stead of a single Kalman Filter. Following the addition of a new track hypothesis, the tracks are divided into clusters where all tracks which share measurements from the N-latest scan are in one cluster. The clusters can then be analyzed as standalone global problems to find the beast possible combination of (possibly mutual exclusive) measurement associations. Linear programming (LP)- and Integer Linear Programming (ILP)-based methods as proposed by \cite{Storms2003} can be used to find the best combinations of newly created track hypotheses in accordance to the assumption that a measurement only can be assigned to one target and that one target can maximally create one measurement.

To limit the size of the track hypothesis tree, the unused edges of the track hypothesis tree will be removed after N scans.

